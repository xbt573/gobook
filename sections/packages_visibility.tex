\subsection{Видимость имён в пакете}
Имена в пакете могут быть внутренними, и публичными, видимость определяется первой буквой в названии, если она большая - имя публично, если она маленькая - имя доступно только внутри пакета.
\begin{minted}[samepage]{go}
package hello

// Публичное имя, видно другим пакетам
func Hello(){}

// Внутреннее имя, видно только внутри этого пакета
func hello(){}
\end{minted}
\subsection{Комментарии}
Комментарии это текст оставленный разработчиками, зачастую описывающий принцип работы или предназначение чего либо. Комментарии начинаются с \verb|//| и игнорируются компилятором.
В Go, комментарии оставленные рядом с функциями, типами и др. являются документационными, и показываются в документации пакета
\begin{minted}[samepage]{go}
// Это документационный комментарий пакета
package hello

import "fmt"

// Документационный комментарий функции
func Hello() {
    fmt.Println("Hello World!")
}
\end{minted}
