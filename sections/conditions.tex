\subsection{Условия}
В Go есть три вида условий:
\begin{itemize}
    \item If
    \item Switch
    \item Select (о нём в отдельной главе)
\end{itemize}

\verb|if| в Go практически ничем не отличается от других языков программирования:
\begin{minted}[samepage]{go}
if 2 * 2 == 5 {
    fmt.Println("Невозможно :O") // никогда не выполнится
}
\end{minted}
\noindent Вокруг \verb|if| тоже не ставятся скобки, как и в \verb|for|.
\noindent Также в \verb|if| можно использовать короткие выражения перед проверкой условия, как в инициализационной части \verb|for|:
\begin{minted}[samepage]{go}
if x := 2 * 2; x == 5 {
    fmt.Println("Невозможно :O") // никогда не выполнится
} else if x := 2 + 2; x == 5 {
    fmt.Println("Невозможно :O") // тоже никогда не выполнится
} else {
    fmt.Println("2 * 2 = 5")
    fmt.Println("2 + 2 = 5")
}
\end{minted}
\noindent Если вы используете это короткое выражение то \verb|;| обязателен для их разделения.

\verb|switch| тоже практически ничем не отличается, за исключением коротких выражений:
\begin{minted}[samepage]{go}
switch x := 2 * 2; x {
case 5:
    fmt.Println("Невозможно :O") // никогда не выполнится

case 2:
    fmt.Println("2 * 2 = 5")
}
\end{minted}
\noindent Важно, что в конце кейса не нужно указывать break.
