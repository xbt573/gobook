\subsection{Пакеты}
Пакеты это механизм разделения кода, обычно по разным категориям.
К примеру, пакет \verb|fmt| отвечает за форматирование и вывод текста в терминал.

Пакет \verb|main| является особенным, он объявляет точку входа в программу/
Что бы импортировать пакет используется директива \verb|import|
\begin{minted}[samepage]{go}
import "fmt"
\end{minted}
Все экспортированые функции, типы, переменные и константы доступны через идентификатор \verb|fmt|.

Пакет должен находиться в директории с таким же именем, иначе компилятор Go посчитает его ненужным и выбросит из процесса компиляции :)

Инициализируем проект с названием \verb|gotour|
\begin{verbatim}
$ go mod init gotour
go: creating new go.mod: module gotour
\end{verbatim}

Создадим папку \verb|hello|, и в ней файл \verb|hello.go|
\begin{minted}[samepage]{go}
package hello

import "fmt"

// Функция Hello выводит "Hello World!" в stdout
func Hello() {
    fmt.Println("Hello World!")
}
\end{minted}

Импортируем этот пакет в \verb|main.go| и вызовем функцию \verb|Hello|
\begin{minted}[samepage]{go}
package main

import "gotour/hello"

func main() {
    hello.Hello()
}
\end{minted}

Скомпилируем и запустим нашу программу
\begin{verbatim}
$ go build
$ ./gotour
Hello World!
\end{verbatim}
