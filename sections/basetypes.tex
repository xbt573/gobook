\subsection{Базовые типы данных}
В Go есть 19 типов данных:
\begin{itemize}
    \item bool - Означает истину или ложь в зависимости от значения
    \item string - Строка
    \item int, int8, int16, int32, int64 - Целочисленные значения разной битности
    \item uint, uint8, uint16, uint32, uint64, uintptr - Целочисленные значения не имеющие знака (всегда больше 0)
    \item byte - идентичен uint8, создан что бы не путать ASCII с целочисленными значениями
    \item rune - идентичен int32, создан с той же целью что и byte, только использует UTF-8
    \item float32, float64 - Числа с плавающей точкой
    \item complex64, complex128 - Комплексные числа
\end{itemize}

Размер \verb|int| зависит от платформы (32 или 64 бита), вам не нужно использовать другие его битности без необходимости
