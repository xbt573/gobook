\subsection{Переменные}
Переменные в Go можно определить двумя способами:
\begin{itemize}
    \item Длинным
    \item Коротким
\end{itemize}

Длинные декларации состоят из ключевого слова \verb|var|, имён переменных, необязательных типов переменных если есть инициализация, и опциональной инициализации
\begin{minted}[samepage]{go}
    var a int
    var d = 5
    var x, y int
    var b, c int = 5, 10
\end{minted}


Короткие декларации состоят из имён переменных, оператора \verb|:=|, и значений. Короткие декларации должны инициализироваться сразу, и могут создаваться только на уровне функций и методов.
\begin{minted}[samepage]{go}
x := 1
a, b, c := 1, 2, true
\end{minted}

Константы определяются длинные переменные, только с использованием ключевого слова \verb|const|
\begin{minted}[samepage]{go}
const a int = 573
const b = 5.73
\end{minted}
